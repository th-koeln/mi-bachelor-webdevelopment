%Da Latex für englischsprachige Texte ausgerichtet ist,
%wird als Dokumentenklasse das "`scrbook"' von Markus Kohm verwendet.
%Dieses ist für deutschsprachige Texte ausgelegt.
%BCOR12mm: 12mm Bindekorrektur (Verbreiterung des linken Randes)
%DIV11: entspricht in etwas der geforderten Textgröße und Seitenränder
%titlepage: eine Titelseite wird verwendet
%a4paper: DIN A4
%oneside: für eine spätere einseitige Bedruckung 
\documentclass[BCOR12mm,DIV11,titlepage,a4paper,oneside,10pt]{scrbook}


% Farben definieren
\usepackage{color}
\usepackage[html]{xcolor}
\definecolor{m_green}{HTML}{00AD2F}
%\definecolor{m_pink}{HTML}{D40B6F}
\definecolor{m_pink}{HTML}{dd1166}
\definecolor{m_grey}{HTML}{555555}
\definecolor{m_lila}{HTML}{9313ce}
\definecolor{m_blau}{HTML}{4952e1}


%Paket für deutsche Silbentrennung etc.
\usepackage{ngerman}

%Paket für Zeichenkodierung, entspricht UTF-8
\usepackage[utf8x]{inputenc}

%Paket das die Ausgabefonts definiert
\usepackage[T1]{fontenc}

%Paket für das Einbinden von Grafiken über die figure-Umgebung
\usepackage{graphicx}

%Paket zum \UTF{0192}ndern der Kopf- und Fußzeile
\usepackage{fancyhdr}
%Benutzt das Paket für eigenen Seitenstil
\pagestyle{fancy} 
%Erzeugt eine Linie in der Kopfzeile (lässt sich mit 0.0pt ausblenden)
\renewcommand*{\headrulewidth}{0.1pt} 
\renewcommand{\headrule}{\hbox to\headwidth{%
  \color{m_pink}\leaders\hrule height \headrulewidth\hfill}}
\lhead{} %Kopfzeile links
%\chead{\thepage} %Kopfzeile mitte
\chead{\includegraphics[height=20pt]{../assets/box.pdf}}
\rhead{} %Kopfzeile rechts
\lfoot{} %Fußzeile links
\cfoot{} %Fußzeile mitte
\rfoot{} %Fußzeile rechts
\makeatother


\newcommand{\changefont}{%
    \fontsize{9}{11}\selectfont
}

  
%\setlength\headheight{40pt}
%\setlength\footheight{20pt}
%\lfoot{\includegraphics[height=40pt]{../assets/box.pdf}}
\rfoot{\thepage}
\lfoot{\changefont Selbstbericht Medieninformatik\\TH Köln // Institut für Informatik\\ \today}


%\UTF{0192}ndert die Seitennummerierung beim Inhaltsverzeichnis mit eigenem Stil
\renewcommand*{\indexpagestyle}{fancy}
%Verhindert die Seitennummerierung auf den Part-Seiten
\renewcommand*{\partpagestyle}{empty}
%\UTF{0192}ndert die Seitennummerierung bei Chapter mit eigenem Stil
\renewcommand*{\chapterpagestyle}{fancy}

%Abbildungsnummerierung ändern (abhängig von chapter, z.B. Abbildung 1.1)
\renewcommand*{\thefigure}{\thechapter.\arabic{figure}}
%Tabellennummerierung ändern (abhängig von chapter, z.B. Tabelle 1.1)
\renewcommand*{\thetable}{\thechapter.\arabic{table}}

%Paket, um ein Glossar/Abkürzungsverzeichnis anzulegen
\usepackage{nomencl}
\let\abbrev\nomenclature
%Der Name wird in Glossar geändert
\renewcommand{\nomname}{Glossar}
%Definiert die Aufteilung im Glossar zwischen Begriffen und Erläuterung
\setlength{\nomlabelwidth}{.25\hsize}
%Definiert die Punktelinien im Glossar
\renewcommand{\nomlabel}[1]{#1 \dotfill}
\setlength{\nomitemsep}{-\parsep}
%Veranlasst die Erstellung des Glossars
\makenomenclature

%Einrückungen nach Absätzen und Grafiken verhindern
\setlength{\parindent}{0pt}

%Verhindern, dass eine neue Seite für ein einzelnes Wort/Zeile verwendet wird
\clubpenalty = 10000 % schliesst Schusterjungen aus 
\widowpenalty = 10000 % schliesst Hurenkinder aus (keine Beleidigung, sondern wirklich ein Fachbegriff)

%Paket für ein deutsches Literaturverzeichnis
\usepackage{bibgerm}

%Paket für die Verwendung von URLs durch den Befehl \url{}
\usepackage{url}

%Paket für Zeilenabstand (onehalfspace, singlespace)
\usepackage{setspace}

%Paket zur Erzeugung von Anführungszeichen durch \enquote{Text}
\usepackage[ngerman]{babel}
\usepackage[babel, german=quotes]{csquotes}

%Paket für farbigen Text
%black,white,green,red,blue,yellow,cyan,magenta
\usepackage{color}

%Paket für farbigen Hintergrund für Verbatim-Umgebung (Quelltext-Umgebung)
\usepackage{fancyvrb}
\usepackage{verbatim,moreverb}
%Grauton für Quelltext-Umgebung definieren 80% Grau
\definecolor{sourcegray}{gray}{.80}
%Paket für Quelltext-Umgebung
\usepackage{listings}

%Paket für Positionierung der Objekte ohne Float (Verwendungsbsp.: \begin{figure}[H])
\usepackage{float}

%Paket zur Erzeugung von Hyperrefs und PDF Informationen
\usepackage[pdftex,plainpages=false,pdfpagelabels,
            pdftitle={Bachelorarbeit},
            pdfauthor={Vorname Nachname}
            ]{hyperref}