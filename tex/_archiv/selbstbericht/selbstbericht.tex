%styles
%Da Latex für englischsprachige Texte ausgerichtet ist,
%wird als Dokumentenklasse das "`scrbook"' von Markus Kohm verwendet.
%Dieses ist für deutschsprachige Texte ausgelegt.
%BCOR12mm: 12mm Bindekorrektur (Verbreiterung des linken Randes)
%DIV11: entspricht in etwas der geforderten Textgröße und Seitenränder
%titlepage: eine Titelseite wird verwendet
%a4paper: DIN A4
%oneside: für eine spätere einseitige Bedruckung 
\documentclass[BCOR12mm,DIV11,titlepage,a4paper,oneside,10pt]{scrbook}


% Farben definieren
\usepackage{color}
\usepackage[html]{xcolor}
\definecolor{m_green}{HTML}{00AD2F}
%\definecolor{m_pink}{HTML}{D40B6F}
\definecolor{m_pink}{HTML}{dd1166}
\definecolor{m_grey}{HTML}{555555}
\definecolor{m_lila}{HTML}{9313ce}
\definecolor{m_blau}{HTML}{4952e1}


%Paket für deutsche Silbentrennung etc.
\usepackage{ngerman}

%Paket für Zeichenkodierung, entspricht UTF-8
\usepackage[utf8x]{inputenc}

%Paket das die Ausgabefonts definiert
\usepackage[T1]{fontenc}

%Paket für das Einbinden von Grafiken über die figure-Umgebung
\usepackage{graphicx}

%Paket zum \UTF{0192}ndern der Kopf- und Fußzeile
\usepackage{fancyhdr}
%Benutzt das Paket für eigenen Seitenstil
\pagestyle{fancy} 
%Erzeugt eine Linie in der Kopfzeile (lässt sich mit 0.0pt ausblenden)
\renewcommand*{\headrulewidth}{0.1pt} 
\renewcommand{\headrule}{\hbox to\headwidth{%
  \color{m_pink}\leaders\hrule height \headrulewidth\hfill}}
\lhead{} %Kopfzeile links
%\chead{\thepage} %Kopfzeile mitte
\chead{\includegraphics[height=20pt]{../assets/box.pdf}}
\rhead{} %Kopfzeile rechts
\lfoot{} %Fußzeile links
\cfoot{} %Fußzeile mitte
\rfoot{} %Fußzeile rechts
\makeatother


\newcommand{\changefont}{%
    \fontsize{9}{11}\selectfont
}

  
%\setlength\headheight{40pt}
%\setlength\footheight{20pt}
%\lfoot{\includegraphics[height=40pt]{../assets/box.pdf}}
\rfoot{\thepage}
\lfoot{\changefont Selbstbericht Medieninformatik\\TH Köln // Institut für Informatik\\ \today}


%\UTF{0192}ndert die Seitennummerierung beim Inhaltsverzeichnis mit eigenem Stil
\renewcommand*{\indexpagestyle}{fancy}
%Verhindert die Seitennummerierung auf den Part-Seiten
\renewcommand*{\partpagestyle}{empty}
%\UTF{0192}ndert die Seitennummerierung bei Chapter mit eigenem Stil
\renewcommand*{\chapterpagestyle}{fancy}

%Abbildungsnummerierung ändern (abhängig von chapter, z.B. Abbildung 1.1)
\renewcommand*{\thefigure}{\thechapter.\arabic{figure}}
%Tabellennummerierung ändern (abhängig von chapter, z.B. Tabelle 1.1)
\renewcommand*{\thetable}{\thechapter.\arabic{table}}

%Paket, um ein Glossar/Abkürzungsverzeichnis anzulegen
\usepackage{nomencl}
\let\abbrev\nomenclature
%Der Name wird in Glossar geändert
\renewcommand{\nomname}{Glossar}
%Definiert die Aufteilung im Glossar zwischen Begriffen und Erläuterung
\setlength{\nomlabelwidth}{.25\hsize}
%Definiert die Punktelinien im Glossar
\renewcommand{\nomlabel}[1]{#1 \dotfill}
\setlength{\nomitemsep}{-\parsep}
%Veranlasst die Erstellung des Glossars
\makenomenclature

%Einrückungen nach Absätzen und Grafiken verhindern
\setlength{\parindent}{0pt}

%Verhindern, dass eine neue Seite für ein einzelnes Wort/Zeile verwendet wird
\clubpenalty = 10000 % schliesst Schusterjungen aus 
\widowpenalty = 10000 % schliesst Hurenkinder aus (keine Beleidigung, sondern wirklich ein Fachbegriff)

%Paket für ein deutsches Literaturverzeichnis
\usepackage{bibgerm}

%Paket für die Verwendung von URLs durch den Befehl \url{}
\usepackage{url}

%Paket für Zeilenabstand (onehalfspace, singlespace)
\usepackage{setspace}

%Paket zur Erzeugung von Anführungszeichen durch \enquote{Text}
\usepackage[ngerman]{babel}
\usepackage[babel, german=quotes]{csquotes}

%Paket für farbigen Text
%black,white,green,red,blue,yellow,cyan,magenta
\usepackage{color}

%Paket für farbigen Hintergrund für Verbatim-Umgebung (Quelltext-Umgebung)
\usepackage{fancyvrb}
\usepackage{verbatim,moreverb}
%Grauton für Quelltext-Umgebung definieren 80% Grau
\definecolor{sourcegray}{gray}{.80}
%Paket für Quelltext-Umgebung
\usepackage{listings}

%Paket für Positionierung der Objekte ohne Float (Verwendungsbsp.: \begin{figure}[H])
\usepackage{float}

%Paket zur Erzeugung von Hyperrefs und PDF Informationen
\usepackage[pdftex,plainpages=false,pdfpagelabels,
            pdftitle={Bachelorarbeit},
            pdfauthor={Vorname Nachname}
            ]{hyperref}

% packages
%\usepackage[super, ]{natbib}

\usepackage{pdfpages}

%\usepackage[toc]{glossaries}

\providecommand{\tightlist}{%
  \setlength{\itemsep}{0pt}\setlength{\parskip}{0pt}}

\usepackage{tabularx}
\newcolumntype{L}[1]{>{\raggedright\arraybackslash}p{#1}}
\newcolumntype{C}[1]{>{\centering\arraybackslash}p{#1}}
\newcolumntype{R}[1]{>{\raggedleft\arraybackslash}p{#1}}

% URL
\usepackage{hyperref}

% Sonderzeichen
\usepackage{textcomp}
\usepackage{eurosym}

\usepackage{setspace}
\renewcommand{\baselinestretch}{1.2}
\setlength{\parskip}{1em}

\usepackage[sfdefault,light]{roboto}  %% Option 'sfdefault' only if the base font of the document is to be sans serif


\usepackage{titlesec}
\titleformat{\chapter}[display]
  {\normalfont\sffamily\mdseries\color{m_lila}}
  {\chaptertitlename\ \thechapter}{14pt}{\huge}
\titleformat{\section}
  {\normalfont\sffamily\Large\mdseries\color{m_lila}}
  {\thesection}{1em}{}

\usepackage[T1]{fontenc}


\usepackage[font=footnotesize]{caption}

%\usepackage[toc]{glossaries}
%\makeglossaries





% Footer definieren
%\usepackage{fancyhdr}
\makeatletter
\def\footrule{{
  \vskip-\footruleskip\vskip-\footrulewidth
  \color{\footrulecolor}
  \hrule\@width\headwidth\@height
  \footrulewidth\vskip\footruleskip
}}
\makeatother
\renewcommand{\footrulewidth}{0.1pt}
\newcommand{\footrulecolor}{m_green}



\begin{document}

% seitennummerierung
%\frontmatter
%\setcounter{page}{3}
%\pagenumbering{arabic}

%Titelseite
%%!TEX root = ../PP.tex

\begin{titlepage}

\begin{center}

%Logo der Fachhochschule Köln
\begin{figure}[!ht]
		\includegraphics[width=0.2\textwidth]{grafiken/logoheader.jpg}
\end{figure}

\vspace{1.3cm}

%Deutscher Titel
\begin{rmfamily}
\textbf{\small „Eine dramatische Entwertung\\kreativer (Audio)Werke hat stattgefunden“ \cite{K01}\\
				Klaus Warstat}\\
\vspace{1.7cm}
\textbf{\huge Digitales}\\
\vspace{0.3cm}
\textbf{\huge \textcolor{m_pink}{\textbf{KLANG}}BUCH}\\
\vspace{0.2cm}
\textbf{\small Musik als Erlebnis}\\
\vspace{0.8cm}
\LARGE Konzeption eines Leitfadens zu Interaktionen und Inhalten eines digitalen Klang-Buchs
\normalsize
\end{rmfamily}


\vspace{1cm}

%Bachelorarbeit 
\begin{LARGE}
\begin{scshape}
\textbf{\huge Bachelorarbeit}\\[0.5em]
\end{scshape}
\end{LARGE}

\vspace{0.7cm}

%im Studiengang...
\begin{large}
vorgelegt an der Technischen Hochschule Köln\\ 
\vspace{0.2cm}
Campus Gummersbach
\end{large}

\vspace{0.7cm}

%ausgearbeitet von...
\begin{large}
ausgearbeitet von\\ 
\vspace{0.2cm}
\begin{LARGE}
Sahrah El ghammaz\\
\end{LARGE}
\end{large}


\vspace{0.4cm}

%im Studiengang...
\begin{large}
im Studiengang\\ 
\vspace{0.2cm}
\textsc{Medieninformatik}
\end{large}

\vspace{0.6cm}

%Autor der Bachelorarbeit und die Prüfer
\begin{tabular}{rl}
        Erster Prüfer:  &  Prof. Dr. Köhler\\
        %Betreuer:  &  Prof. Dr. Lutz Köhler\\
       							    &  \small Technische Hochschule Köln \\[1.0em]
       Zweiter Prüfer:  &  Prof. Kornacher\\
       							    &  \small Technische Hochschule Köln
\end{tabular}

\vspace{0.6cm}

%Ort, Monat der Abgabe
\begin{large}
Marienheide, September 2015
\end{large}

\end{center}



\end{titlepage}

%\onehalfspacing

%%!TEX root = ../PP.tex

%TODO: 




\chapter{Notizen}\label{einleitung}

Ideenbuch, in dem mögliche Funktionen definiert sind


Audio\\
Multitrack, einzelne Spuren stummschalten\\
Remixe\\
Versionen eines Songs, die ab bestimmten Punkten in eine andere Richtung gehen\\
Songtexte in einer anderen Sprache\\
selbst mixen\\
equalizer\\
Songs ineinander mixen\\
großflächiges Feld, das man berühren kann um Klang zu erzeugen\\
Klavier\\

Visuell\\
Elemente die leuchten\\
Animationen\\
Videos\\

haptische Elemente\\

Buch hat narrativen Charakter
Multivisual
Ideensammlung von Funktionen. Soll auch die musikalische Vielfalt ausdrücken.






 









%Kurzfassung
%%!TEX root = ../PP.tex

\chapter*{Kurzfassung}

Der Umgang mit und die Wertschätzung von Musik hat sich seit der Schallplatte bis zu den Streamingdiensten der heutigen Zeit stark verändert. Der Gesamtumsatz der Musik hat, seit Einführung der MP3, abgenommen, während der Konsum von Musik angestiegen ist.\cite{Musik2014} Musik ist heutzutage ein allgegenwärtiges, leicht zugängliches Konsumgut. Die abnehmende Wertschätzung ist ein Trend, der von Musikern, Künstlern und Musikfans gleichermaßen als Verlust bewertet wird.\cite{K01}\cite{Lost}\\

Eine Idee, um diesem Trend etwas entgegen zu setzen, ist das Schaffen eines neuen Mediums - das digitale Klangbuch. Das Konzept des digitalen Klangbuchs sieht ein großformatiges, physisches Buch als erweitertes, interaktives \gls{artwork} und sinnliches Gesamtpaket vor. Es soll Musik, Bilder, Grafiken und Informationen zum Musiker / Künstler und zur Musik enthalten und den Hörer / Fan interaktiv einbinden, fesseln und begeistern. Mit dem digitalen Klangbuch sollen mehrere Sinne gleichzeitig angesprochen werden: Der akustische, visuelle und der haptische Sinn.\\

Die Entwicklung des digitalen Klangbuchs ist ein Projekt, das aus vielen Teilprojekten besteht. Eines dieser Teilprojekte wurde im vorangegangen Praxisprojekt mit dem Thema „Konzeption eines erweiterten Buches als Tonträger zur Verknüpfung von Audio und Interaktion zur Aufwertung kreativer Audiowerke“ erarbeitet. Ziel des Praxisprojekt war die Beantwortung der Frage: Was sind geeignete technologische Ansätze um ein digitales Klangbuch zu entwickeln?\\

Die Bachelorarbeit setzt das vorangegangene Praxisprojekt fort.\\
Das digitale Klangbuch und dessen Möglichkeiten sind für außenstehende Personen nicht einfach zu verstehen. Ziel der Bachelorarbeit ist die Beantwortung der folgenden Frage: Wie lassen sich die Möglichkeiten und Grenzen des digitalen Klangbuchs gegenüber Musikern und Künstlern vermitteln? Zur Beantwortung dieser Frage wurden im ersten Schritt die Möglichkeiten und Grenzen des digitalen Klangbuchs erläutert. Daraufhin wurden Darstellungs- und Funktionsmöglichkeiten des Klangbuchs erarbeitet. Im Anschluss wurden Arten von Leitfäden recherchiert. Im nächsten Schritt wurde ein Leitfaden konzipiert und entwickelt: Der Aufbau des Leitfadens wurde diskutiert, Layouts gestaltet, Audiomaterial produziert und \gls{videoclip}s konzipiert, gedreht, montiert und vertont. Im Anschluss wurde ein Videokanal auf YouTube erstellt und eine Webseite entwickelt.\\

Die Arbeit wurde mit einem Fazit und einem Ausblick abgeschlossen.


%\singlespacing

%Inhaltsverzeichnis
%%!TEX root = ../PP.tex

\tableofcontents
%\onehalfspacing

%=== Hauptteil =======================================================
%Seitennummerierung des Hauptteils
\mainmatter
%Nummerierung beginnt beim Hauptteil ab Seite 4(muss angepasst werden)
%\setcounter{page}{7}

\chapter{Leitfaden // Programmakkreditierung // Siegel der Stiftung zur
Akkreditierung von Studiengängen in Deutschland (Akkreditierungsrat)}\label{Leitfaden // Programmakkreditierung // Siegel der Stiftung zur
Akkreditierung von Studiengängen in Deutschland (Akkreditierungsrat)}


\section{Vorschläge für die Bearbeitung und
Gliederung}\label{vorschluxe4ge-fuxfcr-die-bearbeitung-und-gliederung}

Die Durchführung eines Akkreditierungsverfahrens basiert auf der Vorlage
eines sogenannten Selbstberichtes seitens der antragstellenden
Hochschule.

Die Phase der Erstellung dieser Selbstbewertung bietet die Möglichkeit,
interne Qualitätssicherungs- (und Reflexions-)Prozesse zu nutzen, um
relevante Interessenträger einzubeziehen und Verbesserungspotentiale
freizusetzen. Im Idealfall wird das Akkreditierungsverfahren als Projekt
zur Qualitätsentwicklung in der Hochschule genutzt und nicht als formale
Prüfroutine durchlaufen.

Die Erstellung der Selbstbewertung setzt sich aus jeweils zwei Schritten
zusammen:

\begin{enumerate}
\def\labelenumi{\arabic{enumi}.}
\tightlist
\item
  \textbf{Selbstbewertung}: Im Selbstbericht bewertet die Hochschule in
  möglichst komprimierter Form selbst, ob und wie für den zu
  akkreditierenden Studiengang/die zu akkreditierenden Studiengänge die
  einzelnen Kriterien erfüllt sind und welche Besonderheiten ggf. zu
  berücksichtigen sind. Auch Abweichungen von den Kriterien können hier
  erläutert werden.
\end{enumerate}

Der Fokus liegt dabei vorrangig auf einer bewertenden, nicht
deskriptiven Einschätzung, die z.~B. nach Stärken, Schwächen,
Herausforderungen und Lösungen gegliedert sein kann. Die nachstehenden
„Leitfragen`` zu jedem Kriterium sollen dazu \emph{eine Hilfestellung}
bieten.

Die Selbstbewertung ist zugleich ein „Wegweiser`` durch ergänzende
Anhänge. Häufig reichen eine prägnante, kurz gefasste Einschätzung zu
dem jeweiligen Kriterium und ein Verweis auf einen Beleg im Anhang als
Dokumentationsgrundlage für das Akkreditierungsverfahren aus.

Richtet sich der Akkreditierungsantrag auf ein „Cluster`` inhaltlich
verwandter Studiengänge, sollten Informationen, die für alle
Studiengänge des Clusters gleichermaßen gelten, zusammengefasst werden.
Zugleich sollten studiengangspezifische Informationen (z. B. angestrebte
Lernergebnisse, Curriculum etc.) unterscheidbar ausgewiesen sein.

\begin{enumerate}
\def\labelenumi{\arabic{enumi}.}
\tightlist
\item
  \textbf{Evidenzen}: Es ist von zentraler Bedeutung, dass die
  vorgelegten Selbstbewertungen nachvollziehbar dokumentiert und durch
  geeignete Belege („Evidenzen``) untermauert werden. Dazu sollte ein
  Anhang mit entsprechenden Belegen („Evidenzen``) zusammengestellt
  werden. Dieser Anhang sammelt die internen Regelungen, Dokumente,
  quantitativen oder qualitativen Daten und Informationen, die bereits
  in der Hochschule vorliegen -- z. B. weil sie im Zuge der internen
  Qualitätssicherung ohnehin produziert werden und deshalb nicht eigens
  für das Akkreditierungsverfahren erstellt werden müssen. Eine
  \emph{exemplarische} Liste potentieller Nachweise, die nach Bedarf
  ergänzt oder abgeändert werden kann, findet sich als Hilfestellung in
  der nachfolgenden Gliederung der Selbstbewertung.
\end{enumerate}

Es empfiehlt sich, für die Erstellung der Selbstbewertung dieses
Gliederungsschema als Vorlage zu nutzen. Das Schema ist nach den
Akkreditierungskriterien aufgebaut und unterscheidet jeweils zwischen
Leitfragen für die Analyse und Hinweisen zu möglicherweise geeigneten
Evidenzen. Beide sind nicht verbindlich, sondern lediglich als
Hilfestellung gedacht.

Selbstbewertung und Evidenzen können je nach Digitalisierungsgrad des
hochschuleigenen Dokumenten- und Datenmanagements grundsätzlich
elektronisch aufbereitet sein, z.~B. auch Zugänge zu spezifischen
Webseiten, Datenbanken o. ä. enthalten. Je nach Bedarf einzelner
Gutachtergruppen bitten wir im konkreten Fall zusätzlich um die
Papierfassung der Antragsunterlagen, wobei mittelfristig der Übergang
zur ausschließlich elektronischen Dokumentation angestrebt wird.

%

\chapter{Qualifikationsziele des Studiengangskonzeptes}\label{Qualifikationsziele des Studiengangskonzeptes}


Das Studiengangskonzept orientiert sich an Qualifikationszielen. Diese
umfassen fachli-che und überfachliche Aspekte und beziehen sich
insbesondere auf die Bereiche

\begin{itemize}
\item
  wissenschaftliche oder künstlerische Befähigung,
\item
  Befähigung, eine qualifizierte Erwerbstätigkeit aufzunehmen,
\item
  Befähigung zum gesellschaftlichen Engagement
\item
  Persönlichkeitsentwicklung
\end{itemize}

\section{Leitfragen}\label{leitfragen}

\begin{itemize}
\item
  An welcher Stelle sind die jeweils im Kriterium genannten
  Kompetenz-Bereiche im Studiengang nach dem Verständnis der Hochschule
  abgebildet?
\item
  Wie wurde das angestrebte Kompetenzprofil des Studiengangs
  (weiter-)entwickelt (Auslöser, Vorgehen, Beteiligungen)?
\item
  Finden die definierten Kompetenzziele für Absolventen des
  Studienprogramms die Zustimmung von Lehrenden und Studierenden?
\item
  Wurde die Stimmigkeit der Lernziele des Studiengangs in den letzten
  Jahren überprüft? Aus welchen Gründen wurden ggf. Anpassungen
  vorgenommen?
\item
  Gibt es Auffälligkeiten bei den qualitativen oder quantitativen
  Daten/Informationen der Hochschule hinsichtlich der Akzeptanz des
  Kompetenzprofils auf dem Arbeitsmarkt?
\end{itemize}

\section{Mögliche Evidenzen}\label{muxf6gliche-evidenzen}

\begin{itemize}
\item
  Dokumente/Stellen, wo die Ziele und Lernergebnisse verankert u.
  veröffentlicht sind, z.B. Ordnungen, Homepage, Diploma Supplement,
  Studienführer
\item
  Interne Unterlagen, aus denen die Einbeziehung der verschiedenen
  Interessenträger hervorgeht, z.B. Vorgaben, Prozessbeschreibungen,
  Befragungsergebnisse, Protokolle
\item
  Ziele-Module-Matrix
\item
  Modulbeschreibungen, wie sie den Lehrenden und Studierenden zur
  Verfügung stehen
\end{itemize}

%

\chapter{Konzeptionelle Einordnung des Studiengangs in das Studiensystem}\label{Konzeptionelle Einordnung des Studiengangs in das Studiensystem}


Der Studiengang entspricht

(1) den Anforderungen des Qualifikationsrahmens für deutsche
Hochschulabschlüsse vom 21.04.2005 in der jeweils gültigen Fassung,

(2) den Anforderungen der Ländergemeinsamen Strukturvorgaben für die
Akkreditierung von Bachelor- und Masterstudiengängen vom 10.10.2003 in
der jeweils gültigen Fassung,

(3) landesspezifischen Strukturvorgaben für die Akkreditierung von
Bachelor- und Masterstudiengängen,

(4) der verbindlichen Auslegung und Zusammenfassung von (1) bis (3)
durch den Akkreditierungsrat.

\section{Leitfragen}\label{leitfragen}

\begin{itemize}
\item
  Inwieweit sehen die für den Studiengang Verantwortlichen die im
  Kriterium genannten Anforderungen (insbesondere ländergemeinsame und
  ggf. landesspezifische Strukturvorgaben) eingehalten? Wo sieht die
  Hochschule Abweichungen und wie sind diese begründet?
\item
  Auf welcher Berechnungsgrundlage fußt die Zuordnung von Kreditpunkten
  zu einzelnen Modulen?
\item
  Sind alle verbindlich vorgeschriebenen Studienbestandteile
  (einschließlich praktischer Studienphasen) kreditiert? Wenn nein,
  warum nicht?
\item
  Sind bei der Vergabe von Abschlusszeugnis und Diploma Supplement an
  die Studierenden Probleme bekannt geworden? Wenn ja, wie wurde darauf
  reagiert?
\end{itemize}

\section{Mögliche Evidenzen}\label{muxf6gliche-evidenzen}

\begin{itemize}
\item
  Studien-/Prüfungsordnung bzw. Zugangssatzung
\item
  Falls nicht in Ordnungen enthalten, ergänzende Dokumente, die
  Studienstruktur und -dauer, ggf. Studiengangsprofile, ggf. Einordnung
  in konsekutive oder weiterbildende Masterstudiengänge, Abschlüsse und
  Abschlussbezeichnungen belegen
\item
  Modulbeschreibungen, wie sie den Lehrenden und Studierenden zur
  Verfügung stehen
\item
  Dokumente, in denen Studienverläufe und deren Organisation geregelt
  sind (z.~B. Studienverlaufspläne)
\item
  Dokumente, die die Kreditpunktezuordnung hochschulweit /
  studiengangbezogen regeln
\item
  exemplarisches Zeugnis je Studiengang
\item
  exemplarisches Diploma Supplement je Studiengang
\item
  exemplarisches Transcript of Records je Studiengang
\end{itemize}

%

\chapter{Studiengangskonzept}\label{Studiengangskonzept}


Das Studiengangskonzept umfasst die Vermittlung von Fachwissen und
fachübergreifendem Wissen sowie von fachlichen, methodischen und
generischen Kompetenzen.

Es ist in der Kombination der einzelnen Module stimmig im Hinblick auf
formulierte Qualifikationsziele aufgebaut und sieht adäquate Lehr- und
Lernformen vor. Gegebenenfalls vorgesehene Praxisanteile werden so
ausgestaltet, dass Leistungspunkte (ECTS) erworben werden können.

Es legt die Zugangsvoraussetzungen und gegebenenfalls ein adäquates
Auswahlverfahren fest sowie Anerkennungsregeln für an anderen
Hochschulen erbrachte Leistungen gemäß der Lissabon Konvention und
außerhochschulisch erbrachte Leistungen. Dabei werden Regelungen zum
Nachteilsausgleich für Studierende mit Behinderung getroffen.
Gegebenenfalls vorgesehene Mobilitätsfenster werden curricular
eingebunden.

Die Studienorganisation gewährleistet die Umsetzung des
Studiengangskonzeptes.

\section{Leitfragen}\label{leitfragen}

\begin{itemize}
\item
  Welchen Beitrag leistet das Curriculum/leisten die einzelnen Module
  aus Sicht der für den Studiengang Verantwortlichen und Beteiligten zum
  Erreichen des angestrebten Kompetenzprofils?
\item
  Hat sich im Zuge eines Abgleichs von angestrebtem Kompetenzprofil und
  Curriculum in den letzten Jahren Anpassungsbedarf ergeben? Welche
  Gründe gab es dafür? Wie wurde reagiert?
\item
  Wie wird erreicht, dass die Module in sich stimmig sind, zueinander
  passen und wo nötig aufeinander aufbauen? Wie reagieren die für einen
  Studiengang Verantwortlichen, wenn einzelne Module sich nicht (mehr)
  in das Gesamtkonzept des Studiengangs einfügen?
\item
  Woran erkennen die für den Studiengang Verantwortlichen, dass die
  Module eines Studiengangs \emph{in ihrer Gesamtheit} das angestrebte
  akademische Niveau tragen?
\item
  Inwieweit tragen die angebotenen Wahlmöglichkeiten im Studiengang zum
  Erreichen des angestrebten Kompetenzprofils bei?
\item
  Woran erkennen die Lehrenden und die für den Studiengang
  Verantwortlichen, dass die gewählten didaktischen Instrumente und
  Methoden das Erreichen der Lernergebnisse des Studiengangs
  unterstützen?
\item
  Können alle Lehrenden die ihrer Meinung nach idealen didaktischen
  Instrumente und Methoden einsetzen? Wenn nein, warum nicht?
\item
  Welche Elemente unterstützen das eigenständige wissenschaftliche
  Arbeiten von Studierenden?
\item
  Erfüllen die ggf. in einem Studiengang vorgesehenen Praxisphasen die
  Erwartungen im Hinblick auf die angestrebten Lernergebnisse?
\item
  Welchen Prinzipien folgt die Hochschule im Umgang mit extern
  erworbenen Leistungen von Studierenden?
\end{itemize}

\section{Mögliche Evidenzen}\label{muxf6gliche-evidenzen}

\begin{itemize}
\item
  Curriculare Übersicht/Studienverlaufsplan, aus der/dem Semesterlage,
  Umfang und studentische Arbeitslast der Module pro Semester
  hervorgehen (ggf. mit Veröffentlichungsort wie z.~B. Homepage,
  Studienführer, Studien- bzw. Prüfungsordnungen) bzw. Dokumente, in
  denen Studienverläufe und deren Organisation geregelt sind
\item
  Dokumente, aus denen die geltenden Regelungen zur
  (Auslands-)Mobilität, Praxisphasen und Anerkennung von an anderen
  Hochschulen / außerhalb der Hochschule erbrachte Leistungen erkennbar
  sind
\item
  Ziele-Module-Matrix
\item
  Modulbeschreibungen, wie sie den Lehrenden und Studierenden zur
  Verfügung stehen
\item
  Dokumente aus dem täglichen Gebrauch an der Hochschule, aus denen das
  vorhandene Didaktik-Konzept hervorgeht
\item
  Einschlägige Ergebnisse interner Befragungen und Evaluationen
\item
  Ggf. Daten zur (Auslands-)Mobilität von Studierenden und zu
  Praxiseinsätzen von Studierenden
\item
  Informationen über die Profile der Bewerber und der zugelassenen
  Studierenden
\end{itemize}

%

\chapter{Studierbarkeit}\label{Studierbarkeit}


Die Studierbarkeit des Studiengangs wird gewährleistet durch:

\begin{itemize}
\item
  die Berücksichtigung der erwarteten Eingangsqualifikationen,
\item
  eine geeignete Studienplangestaltung
\item
  die auf Plausibilität hin überprüfte (bzw. im Falle der
  Erstakkreditierung nach Er-fahrungswerten geschätzte) Angabe der
  studentischen Arbeitsbelastung,
\item
  eine adäquate und belastungsangemessene Prüfungsdichte und
  -organisation,
\item
  entsprechende Betreuungsangebote sowie
\item
  eine fachliche und überfachliche Studienberatung.
\end{itemize}

Die Belange von Studierenden mit Behinderung werden berücksichtigt.

\section{Leitfragen}\label{leitfragen}

\begin{itemize}
\item
  Woran erkennen die Verantwortlichen, dass die (formalen und
  fachlich-inhaltlichen) Zugangskriterien das Erreichen des angestrebten
  Kompetenzprofils unterstützen?
\item
  Ggf.: Wie wurde reagiert, wenn die Zugangsregelungen diesen Zweck aus
  Sicht der für den Studiengang Verantwortlichen nicht erfüllt haben?
\item
  Wie schätzen die für den Studiengang Verantwortlichen und daran
  Beteiligten~-- einschließlich der Studierenden -- die studentische
  Arbeitsbelastung ein? Welche Probleme treten auf? Was wird zu deren
  Lösung unternommen?
\item
  Sind hinsichtlich des Studienabschlusses in der vorgesehenen Zeit in
  den vergangenen Jahren Probleme aufgetreten? Wenn ja, welche? Wie
  wurden sie behandelt?
\item
  Inwieweit sind individuelle Mobilitätsfenster für Studierende im
  Studienverlauf realisierbar? Welche Probleme gibt es? Wie wurde darauf
  reagiert?
\item
  Welche Auswirkungen auf die Studierbarkeit haben die vorhandenen
  (prüfungsrelevanten) Regelungen zu Wiederholungsmöglichkeiten,
  Nachteilsausgleich bei Behinderung, Nichterscheinen im Krankheitsfall
  etc.?
\item
  Gab es Fälle, in denen sich die konkrete Prüfungsorganisation (z.~B.
  Terminierung der Prüfungen, Korrekturzeiten) nachteilig auf den
  Studienverlauf ausgewirkt haben? Wenn ja, welche Konsequenzen wurden
  gezogen?
\item
  Welche der vorhandenen Betreuungs- und Beratungsangebote für
  Studierende halten die für den Studiengang Verantwortlichen und
  Beteiligten -- einschließlich der Studierenden -- für besonders
  effektiv im Hinblick auf den Studienerfolg?
\item
  Welche Betreuungs- und Beratungsangebote für Studierende vermissen die
  für den Studiengang Verantwortlichen und Beteiligten -- einschließlich
  der Studierenden? Warum werden sie nicht realisiert?
\item
  Inwieweit werden Belange von Studierenden mit Behinderung
  berücksichtigt?
\end{itemize}

\section{Mögliche Evidenzen}\label{muxf6gliche-evidenzen}

\begin{itemize}
\item
  Ggf. Zugangssatzung sowie Informationen über die
  Studiengangsvoraussetzungen auf Webseiten, in Studienführern etc.
\item
  Einschlägige Ergebnisse interner Erhebungen und Evaluationen -- ggf.
  Daten zur studentischen Arbeitslast
\item
  Studienverlaufsplan, aus der/dem Semesterlage, Umfang und studentische
  Arbeitslast der Module pro Semester hervorgehen (ggf. mit
  Veröffentlichungsort wie z.~B. Homepage, Studienführer, Studien- bzw.
  Prüfungsordnungen) bzw. Dokumente, in denen Studienverläufe und deren
  Organisation geregelt sind
\item
  Dokumente, aus denen die geltenden Regelungen zur
  (Auslands-)Mobilität, Praxisphasen und Anerkennung von an anderen
  Hochschulen / außerhalb der Hochschule erbrachten Leistungen erkennbar
  sind
\item
  Dokumente aus dem täglichen Gebrauch an der Hochschule, aus denen das
  vorhandene Beratungs- und Betreuungskonzept hervorgeht
\item
  (statistische) Daten zu Studienverläufen
\item
  Ggf. Daten zur (Auslands-)Mobilität von Studierenden und zu
  Praxiseinsätzen von Studierenden
\item
  Ggf. weitere einschlägige Ergebnisse interner Befragungen und
  Evaluationen (auch Auffälligkeiten hinsichtlich der Wirkung von ggf.
  vorhandenen Maßnahmen zur Vermeidung von Ungleichbehandlungen in der
  Hochschule)
\end{itemize}

%

\chapter{Studiengangsbezogene Kooperationen}\label{Studiengangsbezogene Kooperationen}


Beteiligt oder beauftragt die Hochschule andere Organisationen mit der
Durchführung von Teilen des Studiengangs, gewährleistet sie die
Umsetzung und die Qualität des Studiengangskonzeptes. Umfang und Art
bestehender Kooperationen mit anderen Hochschulen, Unternehmen und
sonstigen Einrichtungen sind beschrieben und die der Kooperation zu
Grunde liegenden Vereinbarungen dokumentiert.

\section{Leitfragen}\label{leitfragen}

\begin{itemize}
\tightlist
\item
  Funktionieren die hochschulinternen und hochschulexternen
  Kooperationen aus Sicht der für den Studiengang Verantwortlichen?
\end{itemize}

\section{Mögliche Evidenzen}\label{muxf6gliche-evidenzen}

\begin{itemize}
\tightlist
\item
  Kooperationsverträge, Regeln für interne/externe Kooperationen
\end{itemize}

%



\chapter{Ausstattung}\label{Ausstattung}


Die adäquate Durchführung des Studiengangs ist hinsichtlich der
qualitativen und quantitativen personellen, sächlichen und räumlichen
Ausstattung gesichert. Dabei werden Verflechtungen mit anderen
Studiengängen berücksichtigt. Maßnahmen zur Personalentwicklung und
-qualifizierung sind vorhanden.

\section{Leitfragen}\label{leitfragen}

\begin{itemize}
\item
  Auf welche Weise stellen die für den Studiengang Verantwortlichen
  fest, dass Umfang und fachliche Qualifikation des Lehrpersonals für
  Lehre und Betreuung ausreichen?
\item
  Wie zufrieden sind die am Studiengang Beteiligten mit den Ressourcen
  für Lehre, Betreuung und Administration?
\item
  Wie reagieren die für den Studiengang Verantwortlichen auf auftretende
  Probleme und Engpässe?
\item
  Woran wird die Qualität von ggf. eingesetzten Lehrbeauftragten fest
  gemacht?
\item
  Inwieweit sind Forschungs- und Entwicklungstätigkeiten der Lehrenden
  der Studiengangsentwicklung förderlich?
\item
  Wer ist für die fachliche und didaktische Weiterentwicklung der
  Lehrenden verantwortlich?
\item
  Woran erkennen die Verantwortlichen, dass Weiterbildungsmaßnahmen
  erwünscht oder erforderlich sind?
\item
  Wie zufrieden sind die am Studiengang Beteiligten mit der sächlichen
  Ausstattung?
\item
  Wie reagieren die für den Studiengang Verantwortlichen auf Engpässe in
  der Ausstattung?
\end{itemize}

\section{Mögliche Evidenzen}\label{muxf6gliche-evidenzen}

\begin{itemize}
\item
  Beschreibung des Personals
\item
  Dokument aus dem täglichen Gebrauch der Hochschule, aus dem die
  ausreichende Lehrkapazität hervorgeht
\item
  Anzahl der Studierenden
\item
  Darstellung des didaktischen Weiterbildungsangebotes (ggf. Verweis auf
  Webseite) und von Maßnahmen zur Unterstützung der Lehrenden bei dessen
  Inanspruchnahme
\item
  Daten zu wahrgenommenen Weiterbildungsaktivitäten, z.~B.
  Forschungssemester, Gastprofessuren, Seminare, Tagungen, Workshops
\item
  (Kurz-)Darstellung der studiengangsbezogenen Forschungsaktivitäten
\item
  Dokumente aus dem täglichen Gebrauch der Hochschule, in denen die
  Ausstattung dargestellt wird, z.B. Laborhandbücher, Inventarlisten,
  Finanzpläne
\end{itemize}

%

\chapter{Transparenz und Dokumentation}\label{Transparenz und Dokumentation}


Studiengang, Studienverlauf, Prüfungsanforderungen und
Zugangsvoraussetzungen ein-schließlich der Nachteilsausgleichsregelungen
für Studierende mit Behinderung sind dokumentiert und veröffentlicht.

\section{Leitfragen}\label{leitfragen}

\begin{itemize}
\item
  Wie wird sichergestellt, dass inländische und ausländische Studierende
  ihre Rechte und Pflichten kennen?
\item
  Wer hat die Entscheidungsbefugnis über welche Dokumente?
\end{itemize}

\section{Mögliche Evidenzen}\label{muxf6gliche-evidenzen}

\begin{itemize}
\item
  Vorlage aller relevanten Regelungen zu Studienverlauf, Zugang,
  Studienabschluss, Prüfungen, Qualitätssicherung etc., mit Angabe zum
  Status der Verbindlichkeit
\item
  Verweis auf die Stelle, an der diese veröffentlicht sind, z.B.
  Webseiten
\end{itemize}

%

\chapter{Qualitätssicherung und Weiterentwicklung}\label{Qualitätssicherung und Weiterentwicklung}


Ergebnisse des hochschulinternen Qualitätsmanagements werden bei den
Weiterentwicklungen des Studienganges berücksichtigt. Dabei
berücksichtigt die Hochschule Evaluationsergebnisse, Untersuchungen der
studentischen Arbeitsbelastung, des Studienerfolgs und des
Absolventenverbleibs.

\section{Leitfragen}\label{leitfragen}

\begin{itemize}
\item
  Welche Maßnahmen zur Qualitätsverbesserung in und von Studiengängen
  sind in den zurückliegenden Jahren ergriffen worden?
\item
  Welche Elemente der internen Qualitätskontrolle erweisen sich als
  besonders nützlich für kontinuierliche Verbesserungen in einem
  Studiengang?
\item
  Inwieweit findet der Aspekt „Lernergebnisorientierung`` bei der
  Konzeption und in der Praxis der Qualitätssicherungsinstrumente für
  einen Studiengang Berücksichtigung?
\item
  Wie bewerten Studierende die interne Qualitätskontrolle
  und~-entwicklung ihrer Studiengänge hinsichtlich

  \begin{itemize}
  \item
    ihrer Beteiligung?
  \item
    der Auswirkungen auf ihr Studium?
  \end{itemize}
\item
  Wie bewerten Lehrende und die Leitungsebenen die interne
  Qualitätskontrolle und -entwicklung ihrer Studiengänge hinsichtlich

  \begin{itemize}
  \item
    ihrer Beteiligung?
  \item
    der Unterstützung bei der Lösung von Problemen und Verbesserungen in
    der Lehre?
  \end{itemize}
\end{itemize}

\section{Mögliche Evidenzen}\label{muxf6gliche-evidenzen}

\begin{itemize}
\item
  Interne Regelwerke zum Qualitätsmanagement (Evaluationsordnungen u.ä.)
\item
  Exemplarisches Informationsmaterial über das Qualitätsmanagement und
  seine Ergebnisse, das die Hochschule regelmäßig für die Kommunikation
  nach innen und außen nutzt (z.~B. Link zu spezifischen Webseiten,
  Berichte, Flyer)
\item
  Quantitative und qualitative Daten aus Befragungen, Statistiken zum
  Studienverlauf, Absolventenzahlen und -verbleib u.ä.
\end{itemize}

%



\chapter{Geschlechtergerechtigkeit und Chancengleichheit}\label{Geschlechtergerechtigkeit und Chancengleichheit}


Auf der Ebene des Studiengangs werden die Konzepte der Hochschule zur
Geschlechtergerechtigkeit und zur Förderung der Chancengleichheit von
Studierenden in besonderen Lebenslagen wie beispielsweise Studierende
mit gesundheitlichen Beeinträchtigungen, Studierende mit Kindern,
ausländische Studierende, Studierende mit Migrationshintergrund und/oder
aus sogenannten bildungsfernen Schichten umgesetzt.

\section{Leitfragen}\label{leitfragen}

\begin{itemize}
\tightlist
\item
  Liegen Konzepte der Hochschule zur Geschlechtergerechtigkeit und zur
  Förderung der Chancengleichheit von Studierenden in besonderen
  Lebenslagen vor? Wenn ja welche?
\end{itemize}

\section{Mögliche Evidenzen}\label{muxf6gliche-evidenzen}

\begin{itemize}
\tightlist
\item
  Einschlägige Dokumente aus dem alltäglichen Gebrauch der Hochschule,
  die die ggf. vorhandenen Konzepte und Maßnahmen zeigen
\end{itemize}

%



%%!TEX root = ../PP.tex

\newcommand{\fAusklang}{080_ausklang}

\part{Fazit \& Ausblick}
\textbf{Fazit \& Ausblick - Beschreibung}\\
Dieser Teil schließt die Bachelorarbeit mit einem Fazit, einem Ausblick und offnen Fragen ab.

%!TEX root = ../PP.tex


\chapter{Einführung}\label{einleitung}

Die Bachelorarbeit setzt das vorangegangene Praxisprojekt mit dem Thema „Konzeption eines erweiterten Buches als Tonträger zur Verknüpfung von Audio und Interaktion zur Aufwertung kreativer Audiowerke“ fort.

\section{Das digitale Klangbuch}
Der Umgang mit und die Wertschätzung von Musik\footnote{Musik: Im Sinne von reproduzierbarer Musik auf Tonträgern u. Ä.} hat sich seit der Schallplatte bis zur heutigen Zeit stark verändert.\\

{\centering\textcolor{m_pink}{\textit{"Musik wird nicht mehr wertgeschätzt."}}\\\hfill\begin{tiny}{Klaus Warstat \cite{K01}}\end{tiny}}\\

Musik ist heutzutage ein allgegenwärtiges Konsumgut, das weniger wertgeschätzt wird und die Bereitschaft für Musik zu zahlen ist stark gesunken.\cite{Spotify} Dieser Trend wird von Musikern, Künstlern und Musikfans als Verlust empfunden.\cite{K01}\cite{Lost}\\


Eine Idee, um diesem Trend etwas entgegen zu setzen, ist das Schaffen eines neuen Mediums - das digitale Klangbuch. Das Konzept des digitalen Klangbuchs sieht ein großformatiges, physisches Buch als erweitertes, interaktives \gls{artwork} und sinnliches Gesamtpaket vor. Es soll Musik, Bilder, Grafiken und Informationen zum Künstler und zur Musik enthalten und den Hörer / Fan interaktiv einbinden, fesseln und begeistern.\\ 

{\centering\textcolor{m_pink}{\textit{"Früher war das Musikhören, wie das Telefonieren, ein Ereignis, das den Alltag unterbrach und dem Aufmerksamkeit gewidmet wurde."}}\\\hfill\begin{tiny}{Edo Reents \cite{FAZ1}}\end{tiny}}\\

Das Buch ist in Kapiteln aufgeteilt. Jedes Kapitel ist für eine Audiodatei reserviert und kann mehrere Seiten enthalten. Die Inhalte der einzelnen Seiten sollen zu den zugehörigen Audiodateien einen Bezug haben, z. B. könnte die Entstehungsgeschichte eines \gls{song}s durch passende Bilder und Texte erzählt werden. Das \gls{artwork}, das einem Musikalbum mehr Informationen und Inhalte zum Musiker und Künstler gibt, kann in das Buch großflächig und ausführlich integriert werden. Ziel ist, dass die Verbindung von Audio und Buch bzw. von Audiodatei zu passenden visuellen Inhalten zu einem neuen Erleben von Audio, Bildern und Texten führt.\\ 
Durch Umblättern einer Seite oder dem Berühren von speziell gedruckten Elementen im digitalen Klangbuch, können Aktionen ausgeführt werden - z. B. das Starten eines \gls{song}s. Ein Beispiel, wie der Inhalt und die Funktionen eines digitalen Klangbuchs aussehen könnte, zeigt Abbildung \ref{fig:klangbuch1}.


\begin{figure}[H]
\centering
\includegraphics[width=1.0\textwidth]{grafiken/bsplinien.png}
\caption{Beispiel-Inhalt eines digitalen Klangbuchs: Auf der linken Seite ist ein Bild abgedruckt. Berührungsempfindliche Elemente auf der rechten Seite wurden mit einer speziellen Tinte gedruckt und sind mit einer schwarzen Linie mit dem Buchrücken verbunden. Bei Berührung lösen sie eine Funktion aus. Wird der Würfel (unterhalb der Überschrift) berührt, startet der \gls{song}. Bei wiederholter Berührung wird der \gls{song} gestoppt. Die Lautstärke kann ebenfalls beeinflusst werden (Berührung der Felder über das Plus- und Minuszeichen)\\Foto: Sahrah El ghammaz}
\label{fig:klangbuch1}
\end{figure}

Durch das Hören der Musik, das Betrachten der Bilder, Grafiken und Texte und durch das Berühren / Interagieren mit dem Buch werden mehrere Sinne angesprochen und damit ein besonderes Erlebnis geschaffen. Durch das Klangbuch kann der Fan eine Bindung zu der Musik und dem Künstler aufbauen. Dadurch, dass der Fan ein Buch in den Händen hält, mit dem er interaktiv agieren kann, nimmt er sich mehr Zeit für die Musik. Ein Musikalbum wird mit dem digitalen Klangbuch zu einem Gesamterlebnis aufgewertet.



\section{Das Praxisprojekt}
Die Entwicklung des digitalen Klangbuchs ist ein Projekt, das aus vielen Teilprojekten besteht. Eines dieser Teilprojekte ist die Beantwortung der Frage: Was sind geeignete technologische Ansätze um ein digitales Klangbuch zu entwickeln?\\ 
Ziel des Praxisprojekts war es diese Frage zu beantworten. Dazu wurden verschiedene Technologien recherchiert, teilweise getestet und bewertet. Diese lassen sich grob in folgende Kategorien unterteilen: Digitales und Elektronisches Papier, Leitende Tinte, Gedruckte Elektronik und klassische elektronische Bauteile.\\
Anschließend wurden, auf Basis der Technologien, grobe Realisierungsansätze eines digitalen Klangbuchs entwickelt. 
Die Arbeit konnte mit dem Ergebnis abgeschlossen werden, dass es viele Technologien gibt, mit denen ein digitale Klangbuch entwickelt werden kann. Die recherchierten Technologien können einzeln, aber auch kombiniert eingesetzt werden.\\
Im Ausblick des Praxisprojekt wurden folgende, weitere Teilprojekte zum Gesamtprojekt "Digitales Klangbuch" vorgeschlagen:

\begin{itemize}
\item Recherche Komponenten: Weiter Komponenten, wie z. B. ein Mikrocontroller, müssen recherchiert und getestet werden.
\item Programmierung: Für die Steuerung des Buches ist die Programmierung des Mikrocontrollers notwendig.
\item Funktionaler Prototyp: Ein funktionierender Prototyp, der eine ausgewählte Technologie, die Komponenten und die Programmierung enthält, wird entwickelt.
\item Funktionsleitfaden: Erarbeitung und Gestaltung eines Leitfadens, der die Funktionen des digitalen Klang- Buchs erläutert und veranschaulicht.
\item Gestaltung: Nach Auswahl eines Realisierungsansatzes und in Kooperation eines Künstlers / einer Band wird der Inhalt des Buches passend zu den Audiodateien gestaltet.
\item Gestalteter, funktionaler Prototyp und Auswertung: Ein minimalistischer, jedoch gestalteter und funktionaler Prototyp wird entwickelt und getestet.
\item Finanzierungsmöglichkeiten: Es werden verschiedene Finanzierungsmöglichkeiten zur Entwicklung des Buches untersucht und ausgewertet.
\item Crowdfunding: Es wird eine \gls{crowdfunding}-Kampagne zur Finanzierung der Buchentwicklung entwickelt und gestartet.
\item Anwendungsfälle: Es werden weitere Möglichkeiten / Anwendungsfälle zum Einsatz des Buches ermittelt.
\end{itemize}




%{\centering\textcolor{m_pink}{\textit{"Nach dem Einkauf trugen wir die Platten (...) nach Hause, legten sie andächtig auf den Plattenteller (...) und beschäftigten uns tagelang mit den Alben."}}\\\hfill\begin{tiny}{Klaus Warstat \cite{K01}}\end{tiny}}\\



%%%%%%%%%%%%%%%%%%%%%%%%%%%%%%%%%%%%%%%%%%%%%%%%%%%%%%

\section{Entscheidung Bachelorthema \& Problemstellung}\label{einleitung_problem}
Die Wahl auf eines der Teilprojekte zur Fortführung des Gesamtprojekts "Digitales Klangbuch" wurde aufgrund folgender Kriterien getroffen:\\

1. Das Thema sollte innerhalb des zeitlichen Rahmens der Bachelorarbeit realisierbar sein.\\
2. Die Kosten sollten möglichst gering sein.\\
3. Die Bachelorarbeit soll das Gesamtprojekt "Digitales Klangbuch" voran treiben.\\
4. Die Abhängigkeit von Dritten soll möglichst gering sein.\\

Ein Proof of Concept der Technologien und der Komponenten konnte aufgrund der Kosten und der Verfügbarkeit\footnote{Verfügbarkeit: Im Sinne von Zugriff, Zugänglichkeit für Nicht-Fachkräfte} der Technologien nicht in Betracht gezogen werden. Die Technologie "Leitende Tinte" wurde jedoch bereits im Rahmen einer Vorstudie erfolgreich erprobt. Die übrigen Teilprojekte sind teilweise voneinander abhängig, so dass diese ebenfalls nicht für die Bachelorarbeit geeignet sind.\\

Die Wahl fiel daher auf die Entwicklung eines Leitfadens um die Möglichkeiten und Grenzen des digitalen Klangbuchs zu erläutern und zu veranschaulichen:\\
Das digitale Klangbuch und dessen Möglichkeiten sind für außenstehende Personen nicht einfach zu verstehen. Zum Einen, weil das digitale Klangbuch eine abstrakte und neue Idee ist, die man (noch) nicht berühren oder ausprobieren kann. Aber auch, weil die Technologien, die zum Einsatz kommen könnten, für viele Menschen neu und fremd sind. Ein Künstler / Musiker, der Inhalte für das digitale Klangbuch kreieren möchte, weiß schlechtenfalls nicht, welche Möglichkeiten zur Darstellung und Interaktion seiner Inhalte das Buch bietet. Da das digitale Klangbuch weit über ein normales Album und ein \gls{artwork} hinaus geht, sind dem Künstler / Musiker weder die Vielfalt der Möglichkeiten, noch die Grenzen des Buches bekannt.




%%%%%%%%%%%%%%%%%%%%%%%%%%%%%%%%%%%%%%%%%%%%%%%%%%%%%%
\section{{Zielsetzung der Bachelorarbeit}}\label{einleitung_ziel}
Ein Leitfaden, der die Funktionen des digitalen Klangbuchs erläutert und veranschaulicht, würde dem Interessierten die visuellen, auditiven und haptischen Möglichkeiten zur interaktiven Darstellung und Integration von Text, Grafik, Bild und Audio des digitalen Klangbuchs veranschaulichen und ihn idealerweise inspirieren. Hier ein Beispiel:\\

Eine mögliche Funktion des digitalen Klangbuchs wäre die interaktive Darstellung einer Mehrspuraufnahme. Die einzelnen Spuren eines \gls{song}s werden in das Buch integriert und können vom Hörer stumm geschaltet, oder die Lautstärke der einzelnen Spuren angepasst werden. Der Hörer kann so interaktiv den \gls{song} nach seinen Wünschen verändern und seine eigenen \gls{remix}e erstellen.\\

Die Forschungsfrage der Bachelorarbeit lautet: Wie lassen sich die Möglichkeiten und Grenzen des digitalen Klangbuchs Musikern und Künstlern vermitteln und wie lässt sich ein nichtfunktionaler, real aussehender Prototyp des digitale Klangbuch entwickeln?\\

Neben dem Hauptziel, Musikern und Künstlern die Möglichkeiten und Grenzen des Klangbuchs aufzuzeigen, können zwei weitere Ziele mit dem Vorhaben erreicht werden. Das digitale Klangbuch ist bislang nur ein Konzept. Mit einem Leitfaden, der das digitale Klangbuch und seine Möglichkeiten und Funktionen präsentiert, kann eruiert werden, inwieweit Interesse an einem solchen Konzept besteht. Darüber hinaus ist der Leitfaden ein gutes Hilfsmittel um mit Musikern und Künstlern in den Dialog zu treten.




\section{{Zielgruppe}}\label{einleitung_zielgruppe}
Das primäre Nutzungsszenario des digitalen Klangbuchs liegt im Bereich Musik / Kunst / \gls{artwork}. Die Zielgruppe, die das Buch mit Inhalten (Musik, Kunst, Texte, etc.) füllt, sind Musiker / Künstler. Musiker, die ein Musikalbum mit dem digitalen Klangbuch abbilden möchten und Künstler, die Informationen, Fotografien, Bilder und künstlerische Arbeiten, passend zur Musik des Albums kreieren möchten. Arbeiten Musiker und Künstler zusammen, kann ein Gesamtwerk aus Musik, \gls{artwork} und Interaktion geschaffen werden. An diese Zielgruppe richtet sich der Leitfaden. Die Zielgruppe wird im weiteren Verlauf der Arbeit "KlangKünstler" genannt. Ist eine Differenzierung notwendig, kommen die Begriffe "Musiker" und "Künstler" zum Einsatz. Der Nutzer, der am Ende das fertige Buch in den Händen hält, wird als Hörer oder Fan bezeichnet.\\








%%%%%%%%%%%%%%%%%%%%%%%%%%%%%%%%%%%%%%%%%%%%%%%%%%%%%%
\section{{Vorgehen}}\label{einleitung_vorgehen}

Die Bachelorarbeit ist im Bereich Musik, Kunst und (Web-)Technik platziert. Es werden daher fachspezifische Begriffe verwendet, die im angehängten Glossar erläutert werden.\\

Die Bachelorarbeit lässt sich in folgende Hauptbereiche festlegen, die in den zugehörigen Kapiteln näher erläutert und unterteilt werden:

\begin{itemize}
\item Recherche möglicher Arten von Leitfäden
\item Entscheidung für eine Leitfadenart
\item Ideen und Entwicklung möglicher Funktionen des digitalen Klangbuchs
\item Konzeption eines Leitfadens
\item Umsetzung des Leitfadens
\item Der Leitfaden
\item Fazit
\item Ausblick
\end{itemize}




%%%%%%%%%%%%%%%%%%%%%%%%%%%%%%%%%%%%%%%%%%%%%%%%%%%%%%
\section{{Motivation}}\label{einleitung_motivation}
Die Bachelorarbeit setzt, wie bereits in der Einleitung erklärt, das Praxisprojekt fort. Die Motivation zur Erarbeitung des digitalen Klangbuches in Bezug auf die Bachelorarbeit ist ähnlich der des Praxisprojekts:

"Bei der Themenfindung habe ich versucht die Themen zu kombinieren, die mir wichtig sind - Musik, Kunst, Kreativität und Entwicklung - und diese in einen Bezug zur Medieninformatik zu stellen. Vor allem zur Musik habe ich eine besondere und tiefe Verbindung. Das Thema Musik war schon Bestandteil in verschiedenen vorangegangenen Projekten.\footnote{„Live in Concert“, MCI/MMA Projekt bei Prof. Dr. Hartmann \url{http://bit.ly/XxN0N3}}‘\footnote{„Dynamische Soundanalyse“, WPF Generative Gestaltung bei Prof. Noss \url{http://bit.ly/1RYEc2J}}\\

Wie bereits in der Einleitung erläutert, verlieren Audiowerke durch die einfache Möglichkeit an Vervielfältigung an Wert. Musik ist nicht mehr ein reines Genusswerk, sondern wird konsumiert und weniger wertgeschätzt. Der Hörer, inklusive meiner Person, nimmt sich weniger Zeit und Ruhe um ein Audiowerk zu genießen und mehr über den Künstler und sein Werk zu erfahren. Musik wird nicht mehr als Kunst wertgeschätzt, sondern verkommt zu einem Grundrauschen. Das möchte ich mit diesem Projekt ändern."\cite{pp}\\

Dieses Ziel verfolge ich auch mit dieser Arbeit.\\

Das Praxisprojekt bestand zu einem großen Teil aus der Recherche geeigneter Technologien um ein digitales Klangbuch entwickeln zu können. Die Bachelorarbeit dagegen soll weniger aus Recherche bestehen, sondern durch praktische Auseinandersetzung mit unterschiedlichen Aspekten des Klangbuchs das Gesamtprojekt voran treiben und für Künstler und Musiker greifbar machen.



%\item Geschichte der Tonträger und des \gls{artwork}s
%\item Ideen und Ansätze aus der Buchkunst







 













%%!TEX root = ../PP.tex

\newcommand{\fAusklang}{080_ausklang}

\part{Fazit \& Ausblick}
\textbf{Fazit \& Ausblick - Beschreibung}\\
Dieser Teil schließt die Bachelorarbeit mit einem Fazit, einem Ausblick und offnen Fragen ab.

%!TEX root = ../PP.tex


\chapter{Einführung}\label{einleitung}

Die Bachelorarbeit setzt das vorangegangene Praxisprojekt mit dem Thema „Konzeption eines erweiterten Buches als Tonträger zur Verknüpfung von Audio und Interaktion zur Aufwertung kreativer Audiowerke“ fort.

\section{Das digitale Klangbuch}
Der Umgang mit und die Wertschätzung von Musik\footnote{Musik: Im Sinne von reproduzierbarer Musik auf Tonträgern u. Ä.} hat sich seit der Schallplatte bis zur heutigen Zeit stark verändert.\\

{\centering\textcolor{m_pink}{\textit{"Musik wird nicht mehr wertgeschätzt."}}\\\hfill\begin{tiny}{Klaus Warstat \cite{K01}}\end{tiny}}\\

Musik ist heutzutage ein allgegenwärtiges Konsumgut, das weniger wertgeschätzt wird und die Bereitschaft für Musik zu zahlen ist stark gesunken.\cite{Spotify} Dieser Trend wird von Musikern, Künstlern und Musikfans als Verlust empfunden.\cite{K01}\cite{Lost}\\


Eine Idee, um diesem Trend etwas entgegen zu setzen, ist das Schaffen eines neuen Mediums - das digitale Klangbuch. Das Konzept des digitalen Klangbuchs sieht ein großformatiges, physisches Buch als erweitertes, interaktives \gls{artwork} und sinnliches Gesamtpaket vor. Es soll Musik, Bilder, Grafiken und Informationen zum Künstler und zur Musik enthalten und den Hörer / Fan interaktiv einbinden, fesseln und begeistern.\\ 

{\centering\textcolor{m_pink}{\textit{"Früher war das Musikhören, wie das Telefonieren, ein Ereignis, das den Alltag unterbrach und dem Aufmerksamkeit gewidmet wurde."}}\\\hfill\begin{tiny}{Edo Reents \cite{FAZ1}}\end{tiny}}\\

Das Buch ist in Kapiteln aufgeteilt. Jedes Kapitel ist für eine Audiodatei reserviert und kann mehrere Seiten enthalten. Die Inhalte der einzelnen Seiten sollen zu den zugehörigen Audiodateien einen Bezug haben, z. B. könnte die Entstehungsgeschichte eines \gls{song}s durch passende Bilder und Texte erzählt werden. Das \gls{artwork}, das einem Musikalbum mehr Informationen und Inhalte zum Musiker und Künstler gibt, kann in das Buch großflächig und ausführlich integriert werden. Ziel ist, dass die Verbindung von Audio und Buch bzw. von Audiodatei zu passenden visuellen Inhalten zu einem neuen Erleben von Audio, Bildern und Texten führt.\\ 
Durch Umblättern einer Seite oder dem Berühren von speziell gedruckten Elementen im digitalen Klangbuch, können Aktionen ausgeführt werden - z. B. das Starten eines \gls{song}s. Ein Beispiel, wie der Inhalt und die Funktionen eines digitalen Klangbuchs aussehen könnte, zeigt Abbildung \ref{fig:klangbuch1}.


\begin{figure}[H]
\centering
\includegraphics[width=1.0\textwidth]{grafiken/bsplinien.png}
\caption{Beispiel-Inhalt eines digitalen Klangbuchs: Auf der linken Seite ist ein Bild abgedruckt. Berührungsempfindliche Elemente auf der rechten Seite wurden mit einer speziellen Tinte gedruckt und sind mit einer schwarzen Linie mit dem Buchrücken verbunden. Bei Berührung lösen sie eine Funktion aus. Wird der Würfel (unterhalb der Überschrift) berührt, startet der \gls{song}. Bei wiederholter Berührung wird der \gls{song} gestoppt. Die Lautstärke kann ebenfalls beeinflusst werden (Berührung der Felder über das Plus- und Minuszeichen)\\Foto: Sahrah El ghammaz}
\label{fig:klangbuch1}
\end{figure}

Durch das Hören der Musik, das Betrachten der Bilder, Grafiken und Texte und durch das Berühren / Interagieren mit dem Buch werden mehrere Sinne angesprochen und damit ein besonderes Erlebnis geschaffen. Durch das Klangbuch kann der Fan eine Bindung zu der Musik und dem Künstler aufbauen. Dadurch, dass der Fan ein Buch in den Händen hält, mit dem er interaktiv agieren kann, nimmt er sich mehr Zeit für die Musik. Ein Musikalbum wird mit dem digitalen Klangbuch zu einem Gesamterlebnis aufgewertet.



\section{Das Praxisprojekt}
Die Entwicklung des digitalen Klangbuchs ist ein Projekt, das aus vielen Teilprojekten besteht. Eines dieser Teilprojekte ist die Beantwortung der Frage: Was sind geeignete technologische Ansätze um ein digitales Klangbuch zu entwickeln?\\ 
Ziel des Praxisprojekts war es diese Frage zu beantworten. Dazu wurden verschiedene Technologien recherchiert, teilweise getestet und bewertet. Diese lassen sich grob in folgende Kategorien unterteilen: Digitales und Elektronisches Papier, Leitende Tinte, Gedruckte Elektronik und klassische elektronische Bauteile.\\
Anschließend wurden, auf Basis der Technologien, grobe Realisierungsansätze eines digitalen Klangbuchs entwickelt. 
Die Arbeit konnte mit dem Ergebnis abgeschlossen werden, dass es viele Technologien gibt, mit denen ein digitale Klangbuch entwickelt werden kann. Die recherchierten Technologien können einzeln, aber auch kombiniert eingesetzt werden.\\
Im Ausblick des Praxisprojekt wurden folgende, weitere Teilprojekte zum Gesamtprojekt "Digitales Klangbuch" vorgeschlagen:

\begin{itemize}
\item Recherche Komponenten: Weiter Komponenten, wie z. B. ein Mikrocontroller, müssen recherchiert und getestet werden.
\item Programmierung: Für die Steuerung des Buches ist die Programmierung des Mikrocontrollers notwendig.
\item Funktionaler Prototyp: Ein funktionierender Prototyp, der eine ausgewählte Technologie, die Komponenten und die Programmierung enthält, wird entwickelt.
\item Funktionsleitfaden: Erarbeitung und Gestaltung eines Leitfadens, der die Funktionen des digitalen Klang- Buchs erläutert und veranschaulicht.
\item Gestaltung: Nach Auswahl eines Realisierungsansatzes und in Kooperation eines Künstlers / einer Band wird der Inhalt des Buches passend zu den Audiodateien gestaltet.
\item Gestalteter, funktionaler Prototyp und Auswertung: Ein minimalistischer, jedoch gestalteter und funktionaler Prototyp wird entwickelt und getestet.
\item Finanzierungsmöglichkeiten: Es werden verschiedene Finanzierungsmöglichkeiten zur Entwicklung des Buches untersucht und ausgewertet.
\item Crowdfunding: Es wird eine \gls{crowdfunding}-Kampagne zur Finanzierung der Buchentwicklung entwickelt und gestartet.
\item Anwendungsfälle: Es werden weitere Möglichkeiten / Anwendungsfälle zum Einsatz des Buches ermittelt.
\end{itemize}




%{\centering\textcolor{m_pink}{\textit{"Nach dem Einkauf trugen wir die Platten (...) nach Hause, legten sie andächtig auf den Plattenteller (...) und beschäftigten uns tagelang mit den Alben."}}\\\hfill\begin{tiny}{Klaus Warstat \cite{K01}}\end{tiny}}\\



%%%%%%%%%%%%%%%%%%%%%%%%%%%%%%%%%%%%%%%%%%%%%%%%%%%%%%

\section{Entscheidung Bachelorthema \& Problemstellung}\label{einleitung_problem}
Die Wahl auf eines der Teilprojekte zur Fortführung des Gesamtprojekts "Digitales Klangbuch" wurde aufgrund folgender Kriterien getroffen:\\

1. Das Thema sollte innerhalb des zeitlichen Rahmens der Bachelorarbeit realisierbar sein.\\
2. Die Kosten sollten möglichst gering sein.\\
3. Die Bachelorarbeit soll das Gesamtprojekt "Digitales Klangbuch" voran treiben.\\
4. Die Abhängigkeit von Dritten soll möglichst gering sein.\\

Ein Proof of Concept der Technologien und der Komponenten konnte aufgrund der Kosten und der Verfügbarkeit\footnote{Verfügbarkeit: Im Sinne von Zugriff, Zugänglichkeit für Nicht-Fachkräfte} der Technologien nicht in Betracht gezogen werden. Die Technologie "Leitende Tinte" wurde jedoch bereits im Rahmen einer Vorstudie erfolgreich erprobt. Die übrigen Teilprojekte sind teilweise voneinander abhängig, so dass diese ebenfalls nicht für die Bachelorarbeit geeignet sind.\\

Die Wahl fiel daher auf die Entwicklung eines Leitfadens um die Möglichkeiten und Grenzen des digitalen Klangbuchs zu erläutern und zu veranschaulichen:\\
Das digitale Klangbuch und dessen Möglichkeiten sind für außenstehende Personen nicht einfach zu verstehen. Zum Einen, weil das digitale Klangbuch eine abstrakte und neue Idee ist, die man (noch) nicht berühren oder ausprobieren kann. Aber auch, weil die Technologien, die zum Einsatz kommen könnten, für viele Menschen neu und fremd sind. Ein Künstler / Musiker, der Inhalte für das digitale Klangbuch kreieren möchte, weiß schlechtenfalls nicht, welche Möglichkeiten zur Darstellung und Interaktion seiner Inhalte das Buch bietet. Da das digitale Klangbuch weit über ein normales Album und ein \gls{artwork} hinaus geht, sind dem Künstler / Musiker weder die Vielfalt der Möglichkeiten, noch die Grenzen des Buches bekannt.




%%%%%%%%%%%%%%%%%%%%%%%%%%%%%%%%%%%%%%%%%%%%%%%%%%%%%%
\section{{Zielsetzung der Bachelorarbeit}}\label{einleitung_ziel}
Ein Leitfaden, der die Funktionen des digitalen Klangbuchs erläutert und veranschaulicht, würde dem Interessierten die visuellen, auditiven und haptischen Möglichkeiten zur interaktiven Darstellung und Integration von Text, Grafik, Bild und Audio des digitalen Klangbuchs veranschaulichen und ihn idealerweise inspirieren. Hier ein Beispiel:\\

Eine mögliche Funktion des digitalen Klangbuchs wäre die interaktive Darstellung einer Mehrspuraufnahme. Die einzelnen Spuren eines \gls{song}s werden in das Buch integriert und können vom Hörer stumm geschaltet, oder die Lautstärke der einzelnen Spuren angepasst werden. Der Hörer kann so interaktiv den \gls{song} nach seinen Wünschen verändern und seine eigenen \gls{remix}e erstellen.\\

Die Forschungsfrage der Bachelorarbeit lautet: Wie lassen sich die Möglichkeiten und Grenzen des digitalen Klangbuchs Musikern und Künstlern vermitteln und wie lässt sich ein nichtfunktionaler, real aussehender Prototyp des digitale Klangbuch entwickeln?\\

Neben dem Hauptziel, Musikern und Künstlern die Möglichkeiten und Grenzen des Klangbuchs aufzuzeigen, können zwei weitere Ziele mit dem Vorhaben erreicht werden. Das digitale Klangbuch ist bislang nur ein Konzept. Mit einem Leitfaden, der das digitale Klangbuch und seine Möglichkeiten und Funktionen präsentiert, kann eruiert werden, inwieweit Interesse an einem solchen Konzept besteht. Darüber hinaus ist der Leitfaden ein gutes Hilfsmittel um mit Musikern und Künstlern in den Dialog zu treten.




\section{{Zielgruppe}}\label{einleitung_zielgruppe}
Das primäre Nutzungsszenario des digitalen Klangbuchs liegt im Bereich Musik / Kunst / \gls{artwork}. Die Zielgruppe, die das Buch mit Inhalten (Musik, Kunst, Texte, etc.) füllt, sind Musiker / Künstler. Musiker, die ein Musikalbum mit dem digitalen Klangbuch abbilden möchten und Künstler, die Informationen, Fotografien, Bilder und künstlerische Arbeiten, passend zur Musik des Albums kreieren möchten. Arbeiten Musiker und Künstler zusammen, kann ein Gesamtwerk aus Musik, \gls{artwork} und Interaktion geschaffen werden. An diese Zielgruppe richtet sich der Leitfaden. Die Zielgruppe wird im weiteren Verlauf der Arbeit "KlangKünstler" genannt. Ist eine Differenzierung notwendig, kommen die Begriffe "Musiker" und "Künstler" zum Einsatz. Der Nutzer, der am Ende das fertige Buch in den Händen hält, wird als Hörer oder Fan bezeichnet.\\








%%%%%%%%%%%%%%%%%%%%%%%%%%%%%%%%%%%%%%%%%%%%%%%%%%%%%%
\section{{Vorgehen}}\label{einleitung_vorgehen}

Die Bachelorarbeit ist im Bereich Musik, Kunst und (Web-)Technik platziert. Es werden daher fachspezifische Begriffe verwendet, die im angehängten Glossar erläutert werden.\\

Die Bachelorarbeit lässt sich in folgende Hauptbereiche festlegen, die in den zugehörigen Kapiteln näher erläutert und unterteilt werden:

\begin{itemize}
\item Recherche möglicher Arten von Leitfäden
\item Entscheidung für eine Leitfadenart
\item Ideen und Entwicklung möglicher Funktionen des digitalen Klangbuchs
\item Konzeption eines Leitfadens
\item Umsetzung des Leitfadens
\item Der Leitfaden
\item Fazit
\item Ausblick
\end{itemize}




%%%%%%%%%%%%%%%%%%%%%%%%%%%%%%%%%%%%%%%%%%%%%%%%%%%%%%
\section{{Motivation}}\label{einleitung_motivation}
Die Bachelorarbeit setzt, wie bereits in der Einleitung erklärt, das Praxisprojekt fort. Die Motivation zur Erarbeitung des digitalen Klangbuches in Bezug auf die Bachelorarbeit ist ähnlich der des Praxisprojekts:

"Bei der Themenfindung habe ich versucht die Themen zu kombinieren, die mir wichtig sind - Musik, Kunst, Kreativität und Entwicklung - und diese in einen Bezug zur Medieninformatik zu stellen. Vor allem zur Musik habe ich eine besondere und tiefe Verbindung. Das Thema Musik war schon Bestandteil in verschiedenen vorangegangenen Projekten.\footnote{„Live in Concert“, MCI/MMA Projekt bei Prof. Dr. Hartmann \url{http://bit.ly/XxN0N3}}‘\footnote{„Dynamische Soundanalyse“, WPF Generative Gestaltung bei Prof. Noss \url{http://bit.ly/1RYEc2J}}\\

Wie bereits in der Einleitung erläutert, verlieren Audiowerke durch die einfache Möglichkeit an Vervielfältigung an Wert. Musik ist nicht mehr ein reines Genusswerk, sondern wird konsumiert und weniger wertgeschätzt. Der Hörer, inklusive meiner Person, nimmt sich weniger Zeit und Ruhe um ein Audiowerk zu genießen und mehr über den Künstler und sein Werk zu erfahren. Musik wird nicht mehr als Kunst wertgeschätzt, sondern verkommt zu einem Grundrauschen. Das möchte ich mit diesem Projekt ändern."\cite{pp}\\

Dieses Ziel verfolge ich auch mit dieser Arbeit.\\

Das Praxisprojekt bestand zu einem großen Teil aus der Recherche geeigneter Technologien um ein digitales Klangbuch entwickeln zu können. Die Bachelorarbeit dagegen soll weniger aus Recherche bestehen, sondern durch praktische Auseinandersetzung mit unterschiedlichen Aspekten des Klangbuchs das Gesamtprojekt voran treiben und für Künstler und Musiker greifbar machen.



%\item Geschichte der Tonträger und des \gls{artwork}s
%\item Ideen und Ansätze aus der Buchkunst







 














%%!TEX root = ../PP.tex

\newcommand{\fAusklang}{080_ausklang}

\part{Fazit \& Ausblick}
\textbf{Fazit \& Ausblick - Beschreibung}\\
Dieser Teil schließt die Bachelorarbeit mit einem Fazit, einem Ausblick und offnen Fragen ab.

%!TEX root = ../PP.tex


\chapter{Einführung}\label{einleitung}

Die Bachelorarbeit setzt das vorangegangene Praxisprojekt mit dem Thema „Konzeption eines erweiterten Buches als Tonträger zur Verknüpfung von Audio und Interaktion zur Aufwertung kreativer Audiowerke“ fort.

\section{Das digitale Klangbuch}
Der Umgang mit und die Wertschätzung von Musik\footnote{Musik: Im Sinne von reproduzierbarer Musik auf Tonträgern u. Ä.} hat sich seit der Schallplatte bis zur heutigen Zeit stark verändert.\\

{\centering\textcolor{m_pink}{\textit{"Musik wird nicht mehr wertgeschätzt."}}\\\hfill\begin{tiny}{Klaus Warstat \cite{K01}}\end{tiny}}\\

Musik ist heutzutage ein allgegenwärtiges Konsumgut, das weniger wertgeschätzt wird und die Bereitschaft für Musik zu zahlen ist stark gesunken.\cite{Spotify} Dieser Trend wird von Musikern, Künstlern und Musikfans als Verlust empfunden.\cite{K01}\cite{Lost}\\


Eine Idee, um diesem Trend etwas entgegen zu setzen, ist das Schaffen eines neuen Mediums - das digitale Klangbuch. Das Konzept des digitalen Klangbuchs sieht ein großformatiges, physisches Buch als erweitertes, interaktives \gls{artwork} und sinnliches Gesamtpaket vor. Es soll Musik, Bilder, Grafiken und Informationen zum Künstler und zur Musik enthalten und den Hörer / Fan interaktiv einbinden, fesseln und begeistern.\\ 

{\centering\textcolor{m_pink}{\textit{"Früher war das Musikhören, wie das Telefonieren, ein Ereignis, das den Alltag unterbrach und dem Aufmerksamkeit gewidmet wurde."}}\\\hfill\begin{tiny}{Edo Reents \cite{FAZ1}}\end{tiny}}\\

Das Buch ist in Kapiteln aufgeteilt. Jedes Kapitel ist für eine Audiodatei reserviert und kann mehrere Seiten enthalten. Die Inhalte der einzelnen Seiten sollen zu den zugehörigen Audiodateien einen Bezug haben, z. B. könnte die Entstehungsgeschichte eines \gls{song}s durch passende Bilder und Texte erzählt werden. Das \gls{artwork}, das einem Musikalbum mehr Informationen und Inhalte zum Musiker und Künstler gibt, kann in das Buch großflächig und ausführlich integriert werden. Ziel ist, dass die Verbindung von Audio und Buch bzw. von Audiodatei zu passenden visuellen Inhalten zu einem neuen Erleben von Audio, Bildern und Texten führt.\\ 
Durch Umblättern einer Seite oder dem Berühren von speziell gedruckten Elementen im digitalen Klangbuch, können Aktionen ausgeführt werden - z. B. das Starten eines \gls{song}s. Ein Beispiel, wie der Inhalt und die Funktionen eines digitalen Klangbuchs aussehen könnte, zeigt Abbildung \ref{fig:klangbuch1}.


\begin{figure}[H]
\centering
\includegraphics[width=1.0\textwidth]{grafiken/bsplinien.png}
\caption{Beispiel-Inhalt eines digitalen Klangbuchs: Auf der linken Seite ist ein Bild abgedruckt. Berührungsempfindliche Elemente auf der rechten Seite wurden mit einer speziellen Tinte gedruckt und sind mit einer schwarzen Linie mit dem Buchrücken verbunden. Bei Berührung lösen sie eine Funktion aus. Wird der Würfel (unterhalb der Überschrift) berührt, startet der \gls{song}. Bei wiederholter Berührung wird der \gls{song} gestoppt. Die Lautstärke kann ebenfalls beeinflusst werden (Berührung der Felder über das Plus- und Minuszeichen)\\Foto: Sahrah El ghammaz}
\label{fig:klangbuch1}
\end{figure}

Durch das Hören der Musik, das Betrachten der Bilder, Grafiken und Texte und durch das Berühren / Interagieren mit dem Buch werden mehrere Sinne angesprochen und damit ein besonderes Erlebnis geschaffen. Durch das Klangbuch kann der Fan eine Bindung zu der Musik und dem Künstler aufbauen. Dadurch, dass der Fan ein Buch in den Händen hält, mit dem er interaktiv agieren kann, nimmt er sich mehr Zeit für die Musik. Ein Musikalbum wird mit dem digitalen Klangbuch zu einem Gesamterlebnis aufgewertet.



\section{Das Praxisprojekt}
Die Entwicklung des digitalen Klangbuchs ist ein Projekt, das aus vielen Teilprojekten besteht. Eines dieser Teilprojekte ist die Beantwortung der Frage: Was sind geeignete technologische Ansätze um ein digitales Klangbuch zu entwickeln?\\ 
Ziel des Praxisprojekts war es diese Frage zu beantworten. Dazu wurden verschiedene Technologien recherchiert, teilweise getestet und bewertet. Diese lassen sich grob in folgende Kategorien unterteilen: Digitales und Elektronisches Papier, Leitende Tinte, Gedruckte Elektronik und klassische elektronische Bauteile.\\
Anschließend wurden, auf Basis der Technologien, grobe Realisierungsansätze eines digitalen Klangbuchs entwickelt. 
Die Arbeit konnte mit dem Ergebnis abgeschlossen werden, dass es viele Technologien gibt, mit denen ein digitale Klangbuch entwickelt werden kann. Die recherchierten Technologien können einzeln, aber auch kombiniert eingesetzt werden.\\
Im Ausblick des Praxisprojekt wurden folgende, weitere Teilprojekte zum Gesamtprojekt "Digitales Klangbuch" vorgeschlagen:

\begin{itemize}
\item Recherche Komponenten: Weiter Komponenten, wie z. B. ein Mikrocontroller, müssen recherchiert und getestet werden.
\item Programmierung: Für die Steuerung des Buches ist die Programmierung des Mikrocontrollers notwendig.
\item Funktionaler Prototyp: Ein funktionierender Prototyp, der eine ausgewählte Technologie, die Komponenten und die Programmierung enthält, wird entwickelt.
\item Funktionsleitfaden: Erarbeitung und Gestaltung eines Leitfadens, der die Funktionen des digitalen Klang- Buchs erläutert und veranschaulicht.
\item Gestaltung: Nach Auswahl eines Realisierungsansatzes und in Kooperation eines Künstlers / einer Band wird der Inhalt des Buches passend zu den Audiodateien gestaltet.
\item Gestalteter, funktionaler Prototyp und Auswertung: Ein minimalistischer, jedoch gestalteter und funktionaler Prototyp wird entwickelt und getestet.
\item Finanzierungsmöglichkeiten: Es werden verschiedene Finanzierungsmöglichkeiten zur Entwicklung des Buches untersucht und ausgewertet.
\item Crowdfunding: Es wird eine \gls{crowdfunding}-Kampagne zur Finanzierung der Buchentwicklung entwickelt und gestartet.
\item Anwendungsfälle: Es werden weitere Möglichkeiten / Anwendungsfälle zum Einsatz des Buches ermittelt.
\end{itemize}




%{\centering\textcolor{m_pink}{\textit{"Nach dem Einkauf trugen wir die Platten (...) nach Hause, legten sie andächtig auf den Plattenteller (...) und beschäftigten uns tagelang mit den Alben."}}\\\hfill\begin{tiny}{Klaus Warstat \cite{K01}}\end{tiny}}\\



%%%%%%%%%%%%%%%%%%%%%%%%%%%%%%%%%%%%%%%%%%%%%%%%%%%%%%

\section{Entscheidung Bachelorthema \& Problemstellung}\label{einleitung_problem}
Die Wahl auf eines der Teilprojekte zur Fortführung des Gesamtprojekts "Digitales Klangbuch" wurde aufgrund folgender Kriterien getroffen:\\

1. Das Thema sollte innerhalb des zeitlichen Rahmens der Bachelorarbeit realisierbar sein.\\
2. Die Kosten sollten möglichst gering sein.\\
3. Die Bachelorarbeit soll das Gesamtprojekt "Digitales Klangbuch" voran treiben.\\
4. Die Abhängigkeit von Dritten soll möglichst gering sein.\\

Ein Proof of Concept der Technologien und der Komponenten konnte aufgrund der Kosten und der Verfügbarkeit\footnote{Verfügbarkeit: Im Sinne von Zugriff, Zugänglichkeit für Nicht-Fachkräfte} der Technologien nicht in Betracht gezogen werden. Die Technologie "Leitende Tinte" wurde jedoch bereits im Rahmen einer Vorstudie erfolgreich erprobt. Die übrigen Teilprojekte sind teilweise voneinander abhängig, so dass diese ebenfalls nicht für die Bachelorarbeit geeignet sind.\\

Die Wahl fiel daher auf die Entwicklung eines Leitfadens um die Möglichkeiten und Grenzen des digitalen Klangbuchs zu erläutern und zu veranschaulichen:\\
Das digitale Klangbuch und dessen Möglichkeiten sind für außenstehende Personen nicht einfach zu verstehen. Zum Einen, weil das digitale Klangbuch eine abstrakte und neue Idee ist, die man (noch) nicht berühren oder ausprobieren kann. Aber auch, weil die Technologien, die zum Einsatz kommen könnten, für viele Menschen neu und fremd sind. Ein Künstler / Musiker, der Inhalte für das digitale Klangbuch kreieren möchte, weiß schlechtenfalls nicht, welche Möglichkeiten zur Darstellung und Interaktion seiner Inhalte das Buch bietet. Da das digitale Klangbuch weit über ein normales Album und ein \gls{artwork} hinaus geht, sind dem Künstler / Musiker weder die Vielfalt der Möglichkeiten, noch die Grenzen des Buches bekannt.




%%%%%%%%%%%%%%%%%%%%%%%%%%%%%%%%%%%%%%%%%%%%%%%%%%%%%%
\section{{Zielsetzung der Bachelorarbeit}}\label{einleitung_ziel}
Ein Leitfaden, der die Funktionen des digitalen Klangbuchs erläutert und veranschaulicht, würde dem Interessierten die visuellen, auditiven und haptischen Möglichkeiten zur interaktiven Darstellung und Integration von Text, Grafik, Bild und Audio des digitalen Klangbuchs veranschaulichen und ihn idealerweise inspirieren. Hier ein Beispiel:\\

Eine mögliche Funktion des digitalen Klangbuchs wäre die interaktive Darstellung einer Mehrspuraufnahme. Die einzelnen Spuren eines \gls{song}s werden in das Buch integriert und können vom Hörer stumm geschaltet, oder die Lautstärke der einzelnen Spuren angepasst werden. Der Hörer kann so interaktiv den \gls{song} nach seinen Wünschen verändern und seine eigenen \gls{remix}e erstellen.\\

Die Forschungsfrage der Bachelorarbeit lautet: Wie lassen sich die Möglichkeiten und Grenzen des digitalen Klangbuchs Musikern und Künstlern vermitteln und wie lässt sich ein nichtfunktionaler, real aussehender Prototyp des digitale Klangbuch entwickeln?\\

Neben dem Hauptziel, Musikern und Künstlern die Möglichkeiten und Grenzen des Klangbuchs aufzuzeigen, können zwei weitere Ziele mit dem Vorhaben erreicht werden. Das digitale Klangbuch ist bislang nur ein Konzept. Mit einem Leitfaden, der das digitale Klangbuch und seine Möglichkeiten und Funktionen präsentiert, kann eruiert werden, inwieweit Interesse an einem solchen Konzept besteht. Darüber hinaus ist der Leitfaden ein gutes Hilfsmittel um mit Musikern und Künstlern in den Dialog zu treten.




\section{{Zielgruppe}}\label{einleitung_zielgruppe}
Das primäre Nutzungsszenario des digitalen Klangbuchs liegt im Bereich Musik / Kunst / \gls{artwork}. Die Zielgruppe, die das Buch mit Inhalten (Musik, Kunst, Texte, etc.) füllt, sind Musiker / Künstler. Musiker, die ein Musikalbum mit dem digitalen Klangbuch abbilden möchten und Künstler, die Informationen, Fotografien, Bilder und künstlerische Arbeiten, passend zur Musik des Albums kreieren möchten. Arbeiten Musiker und Künstler zusammen, kann ein Gesamtwerk aus Musik, \gls{artwork} und Interaktion geschaffen werden. An diese Zielgruppe richtet sich der Leitfaden. Die Zielgruppe wird im weiteren Verlauf der Arbeit "KlangKünstler" genannt. Ist eine Differenzierung notwendig, kommen die Begriffe "Musiker" und "Künstler" zum Einsatz. Der Nutzer, der am Ende das fertige Buch in den Händen hält, wird als Hörer oder Fan bezeichnet.\\








%%%%%%%%%%%%%%%%%%%%%%%%%%%%%%%%%%%%%%%%%%%%%%%%%%%%%%
\section{{Vorgehen}}\label{einleitung_vorgehen}

Die Bachelorarbeit ist im Bereich Musik, Kunst und (Web-)Technik platziert. Es werden daher fachspezifische Begriffe verwendet, die im angehängten Glossar erläutert werden.\\

Die Bachelorarbeit lässt sich in folgende Hauptbereiche festlegen, die in den zugehörigen Kapiteln näher erläutert und unterteilt werden:

\begin{itemize}
\item Recherche möglicher Arten von Leitfäden
\item Entscheidung für eine Leitfadenart
\item Ideen und Entwicklung möglicher Funktionen des digitalen Klangbuchs
\item Konzeption eines Leitfadens
\item Umsetzung des Leitfadens
\item Der Leitfaden
\item Fazit
\item Ausblick
\end{itemize}




%%%%%%%%%%%%%%%%%%%%%%%%%%%%%%%%%%%%%%%%%%%%%%%%%%%%%%
\section{{Motivation}}\label{einleitung_motivation}
Die Bachelorarbeit setzt, wie bereits in der Einleitung erklärt, das Praxisprojekt fort. Die Motivation zur Erarbeitung des digitalen Klangbuches in Bezug auf die Bachelorarbeit ist ähnlich der des Praxisprojekts:

"Bei der Themenfindung habe ich versucht die Themen zu kombinieren, die mir wichtig sind - Musik, Kunst, Kreativität und Entwicklung - und diese in einen Bezug zur Medieninformatik zu stellen. Vor allem zur Musik habe ich eine besondere und tiefe Verbindung. Das Thema Musik war schon Bestandteil in verschiedenen vorangegangenen Projekten.\footnote{„Live in Concert“, MCI/MMA Projekt bei Prof. Dr. Hartmann \url{http://bit.ly/XxN0N3}}‘\footnote{„Dynamische Soundanalyse“, WPF Generative Gestaltung bei Prof. Noss \url{http://bit.ly/1RYEc2J}}\\

Wie bereits in der Einleitung erläutert, verlieren Audiowerke durch die einfache Möglichkeit an Vervielfältigung an Wert. Musik ist nicht mehr ein reines Genusswerk, sondern wird konsumiert und weniger wertgeschätzt. Der Hörer, inklusive meiner Person, nimmt sich weniger Zeit und Ruhe um ein Audiowerk zu genießen und mehr über den Künstler und sein Werk zu erfahren. Musik wird nicht mehr als Kunst wertgeschätzt, sondern verkommt zu einem Grundrauschen. Das möchte ich mit diesem Projekt ändern."\cite{pp}\\

Dieses Ziel verfolge ich auch mit dieser Arbeit.\\

Das Praxisprojekt bestand zu einem großen Teil aus der Recherche geeigneter Technologien um ein digitales Klangbuch entwickeln zu können. Die Bachelorarbeit dagegen soll weniger aus Recherche bestehen, sondern durch praktische Auseinandersetzung mit unterschiedlichen Aspekten des Klangbuchs das Gesamtprojekt voran treiben und für Künstler und Musiker greifbar machen.



%\item Geschichte der Tonträger und des \gls{artwork}s
%\item Ideen und Ansätze aus der Buchkunst







 













%Die Zähler für Tabellen und Abbildungen werden zurückgesetzt, damit
%in jedem Kapitel die Nummerierung neu beginnt
%\setcounter{table}{1}
%\setcounter{figure}{1}


%Einbindne der Verzeichnisse
%\singlespacing
%%!TEX root = ../PP.tex
 

  %Erzeugt ein Abbildungsverzeichnis
	\listoffigures
	%F\UTF{00B8}gt die Zeile "`Abbildungsverzeichnis"' als Chapter ins Inhaltsverzeichnis ein
	\addcontentsline{toc}{chapter}{Abbildungsverzeichnis}
	
	%Erzeugt ein Tabellenverzeichnis
	%\listoftables
	%F\UTF{00B8}gt die Zeile "`Tabellenverzeichnis"' als Chapter ins Inhaltsverzeichnis ein
	%\addcontentsline{toc}{chapter}{Tabellenverzeichnis}
	
	%Erzeugt ein Glossar
	\glossarystyle{listgroup}
	\glsaddall
	\printglossary[title=Glossar,toctitle=Glossar]
	%\addcontentsline{toc}{chapter}{Glossar}

		
	%\UTF{0192}ndert den Stil des Literaturverzeichnisses
	%\bibliographystyle{geralpha}
	%\bibliographystyle{alphadin}
	\bibliographystyle{alphadin}

	\bibliography{literatur}
	%F\UTF{00B8}gt die Zeile "`Literaturverzeichnis"' als Chapter ins Inhaltsverzeichnis ein
	\addcontentsline{toc}{chapter}{Literaturverzeichnis}
  	


%=== Schlussteil =====================================================
%%Seitennummerierung für den Anhang
\backmatter
%%Seitennummerierung in r\UTF{02C6}mischen Zahlen
%\pagenumbering{Roman}


%Fügt die Zeile "`Anhang"' als Part ins Inhaltsverzeichnis (toc = table of content) ein
%\addcontentsline{toc}{chapter}{Anhang}

%Einbinden des Anhangs mit sämtlichen Verzeichnissen
%\part*{Anhang}
%\includepdf[pages=-]{100_anhang/stories.pdf}


\begin{figure}[H]
\centering
\caption{Website des digitalen Klangbuchs}
\includegraphics[width=0.6\textwidth]{100_anhang/website1.png}
\includegraphics[width=0.6\textwidth]{100_anhang/website2.png}
\includegraphics[width=0.6\textwidth]{100_anhang/website3.png}
\includegraphics[width=0.6\textwidth]{100_anhang/website4.png}
\end{figure}



\begin{figure}[H]
\centering
\caption{Website des digitalen Klangbuchs, responsiv}
\includegraphics[width=0.4\textwidth]{100_anhang/website_r1.png}
\includegraphics[width=0.4\textwidth]{100_anhang/website_r2.png}
\end{figure}


\begin{figure}[H]
\centering
\caption{Website des digitalen Klangbuchs, auf dem iPhone und dem iPad}
\includegraphics[width=0.5\textwidth]{100_anhang/website-iphone.png}
\includegraphics[width=0.7\textwidth]{100_anhang/website-ipadlandscape.png}
\end{figure}



%Einbinden des Eides
%\onehalfspacing
%%!TEX root = ../PP.tex

\chapter*{Eidesstattliche Erklärung}
\addcontentsline{toc}{chapter}{Eidesstattliche Erklärung}
Ich versichere, die von mir vorgelegte Arbeit selbständig verfasst zu haben.\\ \\
Alle Stellen, die wörtlich oder sinngemäß aus veröffentlichten oder nicht veröffentlichten Arbeiten anderer entnommen sind, habe ich als entnommen kenntlich gemacht. Sämtliche Quellen und Hilfsmittel, die ich für die Arbeit benutzt habe, sind angegeben.\\ \\
Die Arbeit hat mit gleichem Inhalt bzw. in wesentlichen Teilen noch keiner anderen Prüfungsgbehörde vorgelegen.
\vspace{1.5cm}
\\
Marienheide, 30. September 2015
\vspace{3cm}
\\
Sahrah El ghammaz


\end{document}
